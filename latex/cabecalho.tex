\documentclass[a6paper,landscape,10pt]{report}

\usepackage[utf8]{inputenc}
%\usepackage{xltxtra}
\usepackage[portuguese]{babel}
\usepackage[top=1cm,left=.4cm,right=.4cm,bottom=.8cm,footskip=0cm]{geometry} %Define margens e distância do rodapé
\usepackage{titletoc} %Para definir o estilo da lista de conteúdo
\usepackage[toc]{multitoc} %Para lista de conteúdo com 3 colunas
\usepackage{pdfpages} %Para incluir pdfs
\usepackage{fancyhdr} %Para personalizar cabeçalho e rodapé
\usepackage[none]{hyphenat} %Impedir quebra de palavras no fim da linha
\usepackage{hyperref}
\usepackage{xcolor}
\usepackage{afterpage}

\hypersetup{
    colorlinks,
    citecolor=black,
    filecolor=black,
    linkcolor=black,
    urlcolor=black
}



%Fonte padrão Sans Serif, na verdade só pro índice...
\usepackage{PTSansNarrow}
\renewcommand*\familydefault{\sfdefault}
\usepackage[T1]{fontenc}
%\usepackage{Intro}

%Define novo estilo de cabeçalho e rodapé
\fancypagestyle{plain}{%
  \renewcommand{\headrulewidth}{0pt}%Sem linha de cabeçalho
  \fancyfoot[C]{} %Sem nada no centro do rodapé
  \fancyfoot[R]{\fontfamily{phv}\selectfont\Large\textbf{\thepage}} %Número da página grande e em negrito à direita do rodapé
}

\renewcommand*{\multicolumntoc}{3} %Define 3 colunas para a lista de conteúdo

\newcommand{\capa}[1]{\includepdf[pagecommand={\thispagestyle{empty}}]{../../capas/#1.pdf}}

\newcommand{\musica}[1]{\phantomsection\addcontentsline{toc}{section}{#1}}
\newcommand{\capitulo}[1]{\phantomsection\addcontentsline{toc}{chapter}{#1}}

%Define estilo dos capítulos no índice
\titlecontents{chapter}
[.3cm] %espaço à esquerda
{\addvspace{.8em}\bf\contentsmargin{0em}} %formatação do título e espaço vertical antes
{} %formato de número
{} %formato de número sem título
{\vspace*{.4em}} %filler
[] %espaço depois do título

%Define estilo da seção no índice
\titlecontents{section}
[.2cm] %espaço à esquerda
{\small} %formatação do título e espaço vertical antes
{} %formato de número
{\textbf{\contentspage}\hspace{2em}} %formato de número sem título
{\hfill} %filler
[] %espaço depois do título
